\let\orilooseness\looseness
\documentclass[oldcsbabel]{csbulletin}
\usepackage[utf8]{inputenc}
\setcounter{secnumdepth}{3}
\usepackage{color}
\usepackage{truncate}
% \setcounter{page}{8}

\usepackage[implicit=false, hidelinks]{hyperref}

\usepackage[
  backend=biber,
  style=iso-numeric,
  sortlocale=cs,
  autolang=other,
  bibencoding=UTF8,
  maxnames=3,
]{biblatex}
\DefineBibliographyStrings{czech}{
  urlseen = {vid\adddot},
}
\addbibresource{14.bib}

% Hack for hyperref
\mubytein 0
% \usepackage[pdftitle=Sazba odstavcu do textovych oblasti,
%   pdfauthor=Jan Sustek,bookmarks=false]{hyperref}
\usepackage{url}

\DefineShortVerb\|
%\def\soub#1{{\sc#1}}
\usepackage[utf8]{inputenc}

% \def\ts#1{{\tt\char`\\name\char`\{ts#1$i$\char`\}}}
{\catcode`\^=12\gdef\ss{^^}}
\fvset{numbers=left,firstnumber=last,numbersep=3pt,
  xleftmargin=15pt}

\def\p#1{\texttt{\char`\\#1}}
\def\uv#1{\char18 #1\char16 }
\def\={\discretionary{-}{-}{-}}
\let\phi\varphi
\def\eqref#1{{\rm(\ref{#1})}}
\def\emph#1{{\sl#1\/}}
\def\soub#1{{\sf#1}}

\input supp-pdf

\clubpenalty10000 %%%%%%%%%%%%%%%%%%%%%%%%%%%%%%%%%%%%%%%%%%%%%%%%%%%%%%%%%%%%%%%%%%%%%%%%%%%%%%%%%%%%%%%%%%

%%%%%%%%%%%%%%%%%%%%%%%%%%% fancyvrb hack - JS
\makeatletter
\def\FV@ListVSpace{%
  \@topsepadd\medskipamount % \topsep
  % z nejakeho duvodu se uvnitr Verbatim nuluje \topsep
  \if@noparlist\advance\@topsepadd\partopsep\fi
  \if@inlabel
    \vskip\parskip
  \else
    \if@nobreak
      \vskip\parskip
      \clubpenalty\@M
    \else
      \addpenalty\@beginparpenalty
      \@topsep\@topsepadd
      \advance\@topsep\parskip
      \addvspace\@topsep
    \fi
  \fi
  \global\@nobreakfalse
  \global\@inlabelfalse
  \global\@minipagefalse
  \global\@newlistfalse}
\makeatother
%%%%%%%%%%%%%%%%%%%%%%%%%%%

\def\casopis#1{{\sl#1\/}}
\def\clanek#1{{\sl#1\/}}
\def\balicek#1{{\sf#1\/}}
\def\<#1>{$\langle\hbox{\sl#1\/}\rangle$}

\def\basnicka#1#2#3{\par\medskip
  \vbox{\leftskip.33\hsize \rightskip0pt plus 1cm \noindent#1\par
  \leftskip0pt plus1fill \rightskip0pt\relax
  \selectlanguage{english}%
  {\sl#2\/}\par{\sc#3}\par}\medskip}

\def\ctt{{\tt comp.text.tex}}
\def\th{{\tt texhax}}

\def\JSvfil{\vfil}
\def\JSvss{\vskip0ptminus12pt}
\def\JSbreak{\break}

\let\PRESKOC\iffalse
\let\COKSERP\fi

\begin{document}
\title
  {Mělo by to fungovat XIII}
\EnglishTitle{It Might Work XIII}
\author{Peter Wilson}
\podpis{Peter Wilson, \url{herries.press@earthlink.net}\\*
  12 Sovereign Close\\*
  Kenilworth, CV8~1SQ, UK}
%\podpis{Jan Šustek, \url{jan.sustek@osu.cz}\\*
%  Ostravská univerzita, Přírodovědecká fakulta,
%  Katedra matematiky\\*
%  30.~dubna~22,
%  CZ-701~03 Ostrava, Czech Republic}

\maketitle[3pt]

\begin{abstract}
Článek ukazuje několik způsobů, jak lze v~\LaTeX u použít verbatim text uvnitř argumentu některých maker. Dále jsou ukázány možnosti, jak zkrátit sazbu textu na danou šířku či výšku.
\end{abstract}
\klicovaslova: Makra, verbatim argumenty, zkracování textu, \LaTeX

\makeatletter
\def\@thefnmark{}\@footnotetext{Z~anglického originálu
\clanek{Glisterings} \cite{GLISTER14} přeložil Jan Šustek.}
\makeatother

\basnicka{\dots Cloath'd all in glistering coats,\\
    which made a shew\dots}
  {Poems and Fancies}{Margaret Cavendish}

Cílem tohoto
seriálu
%článku
je ukázat čtenáři krátké kousky kódu,
které mohou vyřešit některé z~jeho problémů. Doufám, že situaci
ještě více nezkomplikuji v~důsledku mých chyb.
Opravy, poznámky a návrhy na změny budou vždy vítány.

\basnicka{Sir, I have found you an argument,\\
    but I am not obliged to find you an
    understanding.}
  {\noindent}{Samuel Johnson}

\section{Verbatim argumenty}

Nedávno jsem byl upozorněn na problém, že verbatim text nemůže být použitý jako argument makra. Když například chceme vysázet text v~zarámované \texttt{minipage}, lze to udělat jednoduše použitím prostředí \texttt{minipage} uvnitř argumentu makra \p{fbox}.

\begin{Verbatim}
\fbox{\begin{minipage}{0.97\columnwidth}
    Text uvnitř zarámované minipage
  \end{minipage}}
\end{Verbatim}

\noindent
Toto funguje bez problémů až do doby, kdy uvnitř \texttt{minipage} použijeme verbatim text. Potom dostaneme zmatené chybové zprávy, a to i přesto, že bez použití rámečku vše uvnitř \texttt{minipage} funguje.

\LaTeX\ disponuje makrem \p{newsavebox}, s~jehož pomocí lze uložit vysázený text a ten pak použít prostřednictvím makra \p{usebox}.

\medskip

\newsavebox{\minibox}
\newenvironment{framedminipage}[2][c]{%
  \begin{lrbox}{\minibox}
    \begin{minipage}[#1]{#2}}%
  {\end{minipage}\end{lrbox}
    \noindent\fbox{\usebox{\minibox}}}
\let\orifootnote\footnote

\begin{framedminipage}{0.97\columnwidth}
Zde je uvedena definice prostředí \texttt{framedminipage}, které umožňuje použít verbatim text uvnitř rámečku.%
\stepcounter{footnote}%
\expandafter\xdef\csname @thefnmark\endcsname{\thefootnote}%
\csname @makefnmark\endcsname
\xdef\JSfnmarkI{\csname @thefnmark\endcsname}
\begin{Verbatim}[commandchars=/()]
\newsavebox{\minibox}
\newenvironment{framedminipage}[2][c]{%
  \begin{lrbox}{\minibox}
    \begin{minipage}[#1]{#2}}%
  {\end{minipage}\end{lrbox}
    \noindent\fbox{\usebox{\minibox}}}/label(l5)
\end{Verbatim}
Při sazbě tohoto rámečku mělo prostředí \texttt{minipage} šířku rovnu 97\,\% šířky okolní sazby.
\begin{Verbatim}
\begin{framedminipage}{0.97\columnwidth}
  ...
\end{framedminipage}
\end{Verbatim}
\end{framedminipage}
\makeatletter
\def\@thefnmark{\JSfnmarkI}\@footnotetext{V~původním článku nebyl na řádku~\ref{l5} použit příkaz \p{noindent}, což mělo za následek, že makro \p{fbox} při přechodu z~vertikálního módu zahájilo odstavec a vložilo odstavcovou zarážku, což posunulo celý rámeček doprava. (pozn.~překl.)}
\makeatother

\medskip

Prostředí \texttt{lrbox} je analogií \LaTeX ových maker \p{savebox} a \p{sbox}. Zatímco dovnitř argumentu makra nelze přímo vložit verbatim text, dovnitř těla prostředí jej vložit lze. V~uvedeném kódu se nejprve deklaruje registr \p{minibox} typu box, do nějž lze uložit vysázený text. Dále je definováno prostředí \texttt{framedminipage}, a to podobně jako prostředí \texttt{minipage}, včetně nepovinného argumentu určujícího vertikální umístění sazby. Na začátku prostředí \texttt{framedminipage} se zahájí prostředí \texttt{lrbox} a \texttt{minipage} a na jeho konci se zase obě tato prostředí ukončí. Vysázený obsah \texttt{minipage} se tak uloží do registru \p{minibox}, přičemž si musíme uvědomit, že verbatim text už je uvnitř \p{minibox} vysázený. Poté se obsah registru \p{minibox} jen vloží do makra \p{fbox}.

\TeX book~\cite[str.~363]{tb} definuje makro \p{footnote} tak, aby jeho argument mohl obsahovat verbatim text. Knuth píše, že makro je propracované a používá několik triků. Já tomu makru nerozumím, ale ukážu zde jeho hlavní myšlenku. Pro jednoduchost makro \p{verbtext} pouze vysází svůj argument. Nejsem si jistý, zda jsem správně umístil makra \p{color@...}, protože nic podobného v~původních Knuthových makrech nebylo.

\makeatletter
\long\def\verbtext{\vtintro\futurelet\next\vte@t}
\def\vte@t{\ifcat\bgroup\noexpand\next
    \let\next\vt@@t
  \else \let\next\vt@t\fi \next}
\def\vt@@t{\bgroup\aftergroup\vtend\let\next}
\def\vt@t#1{%
  \color@begingroup
  #1\vtmid
  \color@endgroup}
\let\vtintro\relax
\let\vtmid\relax
\let\vtend\relax
\makeatother

\begin{Verbatim}[commandchars=/()]
\makeatletter
\long\def\verbtext{\vtintro\futurelet\next\vte@t}
\def\vte@t{\ifcat\bgroup\noexpand\next/label(l6)
    \let\next\vt@@t
  \else \let\next\vt@t\fi \next}
\def\vt@@t{\bgroup\aftergroup\vtend\let\next}
\def\vt@t#1{%
  \color@begingroup
  #1\vtmid
  \color@endgroup}
\let\vtintro\relax
\let\vtmid\relax
\let\vtend\relax
\makeatother
\end{Verbatim}

\noindent
Makra \p{vtintro} a \p{vtend} se expandují před a~za argumentem a jejich nadefinováním je možné dělat různé triky. Může se stát, že i~definovat makro \p{vtmid} bude užitečné.%
  \footnote{Na řádku~\ref{l6} se testuje, zda je argument makra \p{verbtext} uzavřen do složených závorek (tj.~za tokenem \p{verbtext} následuje otevírací závorka), nebo zda je argumentem jediný token neuzavřený do složených závorek (v~tomto případě samozřejmě nemůže nastat situace, že by argument obsahoval verbatim text). Na základě toho se pak provede buď makro \p{vt@@t}, nebo makro \p{vt@t}.\par
  V~makru \p{vt@@t} se pomocí \p{bgroup} otevře skupina, která se pak uzavře uživatelovou pravou závorkou. Nastaví se, že po uzavření této skupiny se provede \p{vtend}. Konstrukce \p{let}\p{next} odstraní uživatelovu levou závorku. Poté se běžným způsobem načte a zpracuje vnitřek uživatelových složených závorek.\par
  Makro \p{vt@t} by mělo provést totéž, avšak bez nutnosti práce se skupinami. Pro stejnou funkci jako makro \p{vt@@t} by stačilo definovat \texttt{\char`\\def\char`\\vt@t\char`\#1\char`\{\char`\#1\char`\\vtend\char`\}}.\par
  Autor zřejmě chtěl ukázat, že je možné definovat různé chování makra na základě způsobu jeho zavolání. (pozn.~překl.)}

Jednoduché použití makra \p{verbtext} může být
\begin{Verbatim}
\verbtext{Argument makra \verb-\verbtext- může
  obsahovat \verb-\verb- text.}
\end{Verbatim}
s~očekávaným výsledkem
$$\hbox{\verbtext{Argument makra \verb-\verbtext- může obsahovat \verb-\verb- text.}}$$

Následující kód ukazuje použití maker \p{vtintro} a \p{vtend} pro sazbu textu kapitálkami.

\begin{Verbatim}
\makeatletter
\newcommand*{\fred}[1][\@empty]{Frederick%
  \ifx\@empty #1\else~#1\fi}
\makeatother
\def\vtintro{\begingroup\scshape}
\def\vtend{\endgroup}
\verbtext{Makro \verb-\fred[III]-
  vypíše \uv{\fred[III]}, zatímco makro
  \verb-\fred- vypíše pouze \uv{\fred}.}
\end{Verbatim}

\makeatletter
\newcommand*{\fred}[1][\@empty]{Frederick%
  \ifx\@empty #1\else~#1\fi}
\makeatother
\def\vtintro{\begingroup\scshape}
\def\vtend{\endgroup}
\verbtext{Makro \verb-\fred[III]-
  vypíše \uv{\fred[III]}, zatímco makro
  \verb-\fred- vypíše pouze \uv{\fred}.}

V~tomto konkrétním příkladě by šlo jednodušeji psát
\begin{Verbatim}
{\scshape\verbtext{...}}
\end{Verbatim}
a nemuseli jsme se vůbec zatěžovat makry \p{vtintro} a \p{vtend}. Nicméně se určitě najdou situace, kdy se tato makra budou hodit.

\basnicka{Wickedness is always easier than virtue;\\
    for it takes a short cut to everything.}
  {\noindent}{Samuel Johnson}


\section{Zkracování textu}

Další z~dotazů na \ctt\ se týkal problému, kdy je třeba dlouhý text zkrátit například na dva nebo tři řádky.

Již v~tu dobu existoval balíček \soub{truncate} Donalda Arseneaua~\cite{1}, který umožňuje zkrátit text na danou šířku. Na konci zkráceného textu se vytiskne~\uv{\ldots} (\p{ldots}), případně jiná značka, aby se čtenář dozvěděl, že text ještě pokračuje.
Například
\begin{Verbatim}
\truncate{0.9\columnwidth}{Balíček \textsf{truncate} definuje
  makro pro zkracování textu tak, aby nepřekročil zadanou šířku.}
\end{Verbatim}
\noindent
vysází
$$\hbox{%
\truncate{0.9\columnwidth}{Balíček \textsf{truncate} definuje makro
  pro zkracování textu tak, aby nepřekročil zadanou šířku.}%
  }$$

Nicméně dotaz se týkal vertikální analogie makra \p{truncate}. V~odpovědi~\cite{2} na dotaz Donald definoval makro \p{vtruncate} následovně.
\begin{Verbatim}[commandchars=/()]
\newsavebox\descbox
\newsavebox\partialbox
\newcommand{\vtruncate}[2]{%
  \setbox\descbox\vbox{{#2\par}}%
  \setbox\partialbox\vsplit\descbox to #1\relax
  \vtop{\unvbox\partialbox}%/label(l2)
  % or use
  % \par\unvbox\partialbox/label(l1)
  }
\end{Verbatim}
\noindent
Makro má dva argumenty. Druhým argumentem je text, který se zkrátí na výšku danou prvním argumentem.

V~další odpovědi~\cite{3} Will Robertson definoval prostředí \texttt{cutlines}.
\begin{Verbatim}
\makeatletter
\newbox\cut@desc
\newenvironment{cutlines}[1][2]{%
  \@tempcnta=#1\relax
  \setbox\cut@desc\vbox\bgroup
  \parskip=0pt}{%
  \egroup
  \vsplit\cut@desc to \@tempcnta\baselineskip}
\makeatother
\end{Verbatim}
\noindent
Prostředí má jeden nepovinný argument (uzavřený do hranatých závorek), který udává, na kolik řádků se má text uvnitř prostředí zkrátit.

\penalty-1000
Vyzkoušel jsem obě dvě řešení a narazil jsem na následující možné problémy.
\begin{enumerate}
\item Textový argument makra \p{vtruncate} nemůže obsahovat žádný verbatim text. (To nemusí až tolik vadit.)
\item Jestliže v~prostředí \texttt{cutlines} je zadaný počet řádků větší než počet řádků textu, potom je pod text vložena vertikální mezera taková, aby celková výška odpovídala zadanému počtu řádků.
\item V~obou řešeních se stává, že výsledná výška je jiná, než bylo zadáno. Vždy se však liší maximálně o~jeden řádek. Zdá se, že \texttt{cutlines} je přesnější než \p{vtruncate}.
\item Zkrácený text je umístěn ve \p{vbox}u, v~němž nemůže dojít ke stránkovému zlomu.
\end{enumerate}

Po dlouhém ladění jsem dospěl k~definici prostředí \texttt{vcutlines}, které je kombinací obou řešení a které řeší první dva problémy a možná i~třetí problém. Čtvrtý problém se týká všech řešení.\footnote{I~původní článek obsahoval řádek~\ref{l1}. Pokud by se tento řádek použil namísto řádků~\ref{l2} a~\ref{l3}, pak by ke čtvrtému problému nedošlo. (pozn.~překl.)}
Prostředí \texttt{vcutlines} má jeden nepovinný argument, který udává, na jakou výšku se má text uvnitř prostředí zkrátit.


\newsavebox\descbox
\newsavebox\partialbox
\newlength{\vcutl}% for the limit height
\newlength{\Vcutl}% height of full text
\newenvironment{vcutlines}[1][2\baselineskip]{%
  \setlength{\vcutl}{#1}%
  \setbox\descbox\vbox\bgroup
  \parskip=0pt\relax
  }{%
  \egroup
  \Vcutl=\ht\descbox
  \advance\Vcutl \dp\descbox
  \setbox\partialbox\vsplit\descbox to \vcutl\relax
  \vtop{\unvbox\partialbox}
  \ifdim \vcutl<\Vcutl \vtruncont \fi}
\newcommand*{\vtruncont}{\noindent\strut\ldots}

\begin{Verbatim}[commandchars=/()]
\newsavebox\descbox
\newsavebox\partialbox
\newlength{\vcutl}% for the limit height
\newlength{\Vcutl}% height of full text
\newenvironment{vcutlines}[1][2\baselineskip]{%
  \setlength{\vcutl}{#1}%
  \setbox\descbox\vbox\bgroup
  \parskip=0pt\relax
  }{%
  \egroup
  \Vcutl=\ht\descbox
  \advance\Vcutl \dp\descbox
  \setbox\partialbox\vsplit\descbox to \vcutl\relax
  \vtop{\unvbox\partialbox}/label(l3)
  \ifdim \vcutl<\Vcutl \vtruncont \fi}
\newcommand*{\vtruncont}{\noindent\strut\ldots}
\end{Verbatim}

V~následujících příkladech pro testování použijeme text
\begin{Verbatim}
{\itshape Donald Arseneau vytvořil makro \verb-\vtruncate- a
Will Robertson vytvořil prostředí \verb-cutlines-. Obě řešení
zkrátí text, pokud by po vysázení měl větší než zadanou výšku.
V~tomto článku je vytvořeno nové prostředí \verb-vcutlines-,
které je kombinací obou předchozích řešení.}
\end{Verbatim}
\noindent
který obsahuje i~několik verbatim textů.

\penalty-1000

Pro začátek vyzkoušejme prostředí \texttt{vcutlines} s~omezením na 20~řádků (tj.~na výšku \texttt{20}\p{baselineskip}).

\begin{vcutlines}[20\baselineskip]
{\itshape Donald Arseneau vytvořil makro \verb-\vtruncate- a Will Robertson vytvořil prostředí \verb-cutlines-. Obě řešení zkrátí text, pokud by po vysázení měl větší než zadanou výšku. V~tomto článku je vytvořeno nové prostředí \verb-vcutlines-, které je kombinací obou předchozích řešení.}
\end{vcutlines}

\smallskip

A nyní stejný text s~omezením na 3~řádky.

\begin{vcutlines}[3\baselineskip]
{\itshape Donald Arseneau vytvořil makro \verb-\vtruncate- a Will Robertson vytvořil prostředí \verb-cutlines-. Obě řešení zkrátí text, pokud by po vysázení měl větší než zadanou výšku. V~tomto článku je vytvořeno nové prostředí \verb-vcutlines-, které je kombinací obou předchozích řešení.}
\end{vcutlines}

\smallskip

Jestliže dojde ke zkrácení textu, vloží se na konec prostředí makro \p{vtruncont}, které implicitně vloží řádek obsahující pouze~\uv{\ldots} (\p{ldots}). Pro zjištění, zda došlo ke zkrácení textu, se výška celého vysázeného textu porovnává se zadanou výškou.

\renewcommand*{\vtruncont}{}
Makro \p{vtruncont} můžeme například nadefinovat na nic.
\begin{Verbatim}
\renewcommand*{\vtruncont}{}
\end{Verbatim}
\noindent
čímž dostaneme

\begin{vcutlines}[3\baselineskip]
{\itshape Donald Arseneau vytvořil makro \verb-\vtruncate- a Will Robertson vytvořil prostředí \verb-cutlines-. Obě řešení zkrátí text, pokud by po vysázení měl větší než zadanou výšku. V~tomto článku je vytvořeno nové prostředí \verb-vcutlines-, které je kombinací obou předchozích řešení.}
\end{vcutlines}

\smallskip\noindent
V~tom případě se ale čtenář nedozví, zda text ještě pokračuje, nebo ne.

\nocite{*}
\begingroup
\section*{Odkazy}
\sloppy
\AtNextBibliography{\small}
\printbibliography[heading=none]
\endgroup

\begin{summary}
This paper shows several ways in \LaTeX\ how to use verbatim text within argument of some macros. We also show how to truncate text to a given width or height.

\keywords: Macros, verbatim arguments, truncating text, \LaTeX
\end{summary}

\end{document}
%23456789012345678901234567890123456789012345678901234567890123
